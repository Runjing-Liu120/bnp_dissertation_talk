\begin{frame}{Results on STRUCTURE}

We adapt STRUCTURE
{\color{blue} \href{https://web.stanford.edu/group/pritchardlab/publications/pdfs/PritchardEtAl00.pdf}{(Pritchard et al. 2011};
\href{https://www.genetics.org/content/197/2/573}{Raj et al. 2014)}
},
a Bayesian model for population genetics, to include a BNP prior.


We study genetic data from the Taita thrush, an endagered bird species.
The data consists of microsatellites sequences of 155 individiuals at 7 loci.

\begin{figure}[!h]
\centering
\includegraphics[width = 0.8\textwidth]{./figures/structure_example.png}
\caption{The intitial fit at $\alpha = 3$. }
\end{figure}
\end{frame}

\begin{frame}{STRUCTURE: parametric sensitivity}
  \begin{figure}[!h]
    \centering
    \includegraphics[width = \textwidth]{./figures/stucture_alpha_sens.png}
    \caption*{The expected number of posterior in-sample clusters in the thrush data as $\alpha$ varies.}
  \end{figure}

\end{frame}

\begin{frame}{STRUCTURE: functional sensitivity}

  \begin{figure}[!h]
    \centering
    \only<1>{\includegraphics[width = \textwidth]{./figures/structure_mbololo_sens.png}}%
    \only<2>{\includegraphics[width = \textwidth]{./figures/structure_ngangao_sens.png}}%
    \only<3>{\includegraphics[width = \textwidth]{./figures/structure_chawia_sens.png}}
  \end{figure}

\end{frame}

\begin{frame}{Computational times}

\begin{table}[tb]
\centering
\caption*{Compute time of results on the Taita thrush dataset. }
\begin{tabular}{|r|r|}
\hline
    & time (seconds) \\
    \hline
    Initial fit & 7 \\
    \hline
    Hessian solve for $\alpha$ sensitivity & 0.3\\
    Linear approx. $\eta^{lin}(\alpha)$ for $\alpha = 1, ..., 7$ &
        0.006 \\
    Refits $\eta(\alpha)$ for $\alpha = 1, ..., 7$ &
        30 \\
    \hline
    The influence function & 0.6 \\
    Hessian solve for worst-case $\phi$ &
        0.4 \\
    Linear approx. $\eta^{lin}(\epsilon)|_{\epsilon = 1}$
      for worst-case $\phi$ &
        0.001\\
    Refit $\eta(\epsilon)|_{\epsilon = 1}$
      for worst-case $\phi$ &
        10 \\
    \hline
\end{tabular}
\end{table}



\end{frame}

%%%%%%%%%%%%%%%%%%%
% limitations of local sensitivity
%%%%%%%%%%%%%%%%%%%


\begin{frame}{Limitations of local sensitivity}
  \begin{figure}[!h]
    \centering
    \includegraphics[width = \textwidth]{./figures/bad_admix_example.png}
    \caption*{Inferred admixtures after the worst-case perturbation
     to individuals A.
     Individual $n = 26$ had a large increase in admixture proportion of
     population 2 after the refit. }
  \end{figure}

\end{frame}

\begin{frame}{Limitations of local sensitivity}

  \begin{figure}[!h]
    \centering
    \only<1>{\includegraphics[width = \textwidth]{./figures/bad_admix_example_trace0.png}}%
    \only<2>{\includegraphics[width = \textwidth]{./figures/bad_admix_example_trace1.png}}%
  \end{figure}

\end{frame}
