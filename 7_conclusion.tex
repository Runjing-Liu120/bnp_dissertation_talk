\begin{frame}{Conclusions}

\begin{itemize}
\item We provide a tool to efficiently evaluate the sensitivity of the variational posterior to prior chioces.
\item Linearizing the variational parameters provides a reasonable alternative
re-optimizing the variational approximation
after model perturbations.
\item The influence function can provide guidance to find particularly sensitive model perturbations.
\end{itemize}

\end{frame}


\begin{frame}{References}

{\bf A workshop paper: }\newline
Runjing Liu, Ryan Giordano, Michael I. Jordan, Tamara Broderick. \newline
“Evaluating Sensitivity to the Stick Breaking Prior in Bayesian Nonparametrics.”
\newline {\color{blue}\url{https://arxiv.org/pdf/1810.06587.pdf}}

\vspace{1em}

{\bf Code: }\newline
Paragami: parameter folding and flattening for optimization problems \newline
{\color{blue}\url{https://github.com/rgiordan/paragami}}

Vittles: library for sensitivity analysis in optimization problems \newline
{\color{blue}\url{https://pypi.org/project/vittles/}}

JAX: composable transformations of Python+NumPy programs \newline
{\color{blue}\url{https://github.com/google/jax}}


\end{frame}
